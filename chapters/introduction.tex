\chapter{Introduction}

Correlating climate data and biological data has resulted in many interesting findings, especially when concerning the reaction to climatic conditions. This project aims to explore how 23 different bird species respond to the climatic attributes of temperature variations, wind and precipitation variations. The bird species data was provided by Project FeederWatch (PFW), a Cornell Ornithology Lab study in which citizens monitor bird feeders from across the United States and Canada. The climate data that was appended to the PFW data set was provided by Weather Underground (WU). Our hypothesis is that the abundance at the feeders is largely due to the thermoregulation requirements. These requirements are further determined by the body mass of the birds, with smaller birds retaining less heat, thus requiring more food from the feeders to offset the caloric deficit. With respect to wind speed and precipitation variations, we expect to see an overall decline in the number of visits to the feeders as wind speed and precipitation levels increase. This is because flight conditions in these cases are not ideal.

The results indicate that our overall hypothesis in regards to temperature is correct. During colder than average periods of winter, more individuals of the species were spotted at the feeders. This pattern was present throughout most of the 23 species studied in this project. However our hypothesis regarding the body mass is incorrect as there is no evidence of a relationship between bird feeder visits and the thermoregulation needs of the birds in regards to their body mass. Additionally, our hypothesis regarding wind and precipitation is also wrong as the results showed no evidence of the bird species being significantly affected by wind and rain levels. Overall, this project proved the usefulness of correlating climate data with bird feeder data from PFW and thus more research is required to explore other effects. Currently, the largest hurdle for the continuation of this study is the Weather Underground limits on climate data. Overcoming this will allow for more bird species to be studied in regards to the responses the climate conditions.   