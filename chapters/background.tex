

\chapter{Background}

\section{Climate and Biological Responses}

Climate change is increasingly becoming an urgent issue as changing weather patterns have far reaching impacts on biological systems~\cite{baum2009cascading}. Before discussing any further, it is worthwhile to first define the key terms. Climate, in terms of this study, is defined as the prevailing weather conditions of the area. The measurements used to assess the climate for the project include temperature, precipitation and wind speed. There are other features, such as pressure, but that is ignored. Details about this are covered in the later sections. 

Studies correlating climate and biological events provide key information on how animals and plants are reacting to current changes~\cite{bradshaw2006evolutionary, jump2005running}. Additionally, many of these studies offer a glimpse into biological impacts we can expect in the future. The large scale impacts of climate change on animal species can be seen through animal group movements. Changing weather patterns have been listed as one of the top factors in contributing to the decline of large scale animal migrations~\cite{wilcove2008going}. 

Animal sensitivity towards changing climate can also be observed on a smaller scale. For example, Edmun D. Brodie et al. found that garter snakes crawled more slowly, for shorter distances and performed fewer reversals of direction in cooler temperatures~\cite{d1999consistency}. This indicates that garter snakes that are present in regions with cooler climates, thus lower average temperatures, are generally slower than those in warmer climates. Additionally, Canadian red squirrels are breeding earlier in the spring to take advantage of the earlier spruce cone production~\cite{bradshaw2006evolutionary}. These studies indicate that sudden large-scale environmental changes in temperatures may impact every day life of the animals involved.

Animals' response to climate change is also largely determined by the availability of food resources. For birds, which are the focus of this study, these food resources are highly dependent on the surrounding vegetation. The bird species may rely on the plants themselves for food, feeding on fleshy fruits or seeds~\cite{bock1976synchronous}. The birds may also prey upon the insects hiding among the foliage~\cite{holmes1979bird}. 

One study by Robinson et al. suggests that the vegetation affects the arthropod foraging behaviors of birds that can successfully exploit the surrounding habitat~\cite{robinson1982foraging}. Changes in behavior may include perch selection and large-scale habitat selection. However, with climate change it is being observed that plants are in fact migrating at various rates around the globe~\cite{pitelka1997plant}. For example, Cheatgrass invaded western North America for over the past century. Furthermore, this invasion has happened at various increased rates throughout the years rather than at a steady pace~\cite{pitelka1997plant}. This changing vegetation may pose a serious challenge to bird species that are unable to adapt to the rapidly changing habitats.

The aim of this project is to specifically study bird behavior around bird feeders in relation to the following daily climatic factors: mean temperature, maximum temperature, minimum temperature, wind speed and precipitation levels. Feeding birds through bird feeders is a well known practice, but there is little known about consequences of such supplementary food sources~\cite{jones2008feeding}. Large scale surveys, such as Project Feederwatch, provide important data on different bird species' activities at the feeders. This data has the potential to be used in a variety of studies.


\section{Project FeederWatch}

Project FeederWatch (PFW) was developed through the Cornell Lab of Ornithology, and for 31 years enlisted sharp eyed amateur bird watchers to document bird species around various feeders~\cite{louv2012citizen}. For this study, only observations from the years 2007 and through 2012 were used. The citizens will note down important features such as the bird species and number of individuals seen. All observations are made from November through early April~\cite{ProjectO75:online}. The citizen-wide approach allows for hundreds of thousands of observations from across the United States and Canada. However there are drawbacks, the citizens themselves are not formally trained in making and recording observations, thus not all of the submitted data is valid.

To filter out the incorrect data, Project FeederWatch has employed a rigorous validation process that ensures the end data is as accurate as possible. The review procedure involves a range of filtering methods. There are simple approaches, such as cross referencing the observations against a checklist of "allowed" species for each US and Canadian province. Then there are the other more involved approaches. 

One such approach is aimed to solve the complex problem of filtering out incorrect plausible data, such as misidentifying an species as another species that also exists in the region. The proposed solution is to provide educational quizzes/games to the bird watchers so that their bird watching skills can be quantified. This information then can be used to filter out species observations that may be too challenging for the observer to identify~\cite{bonter2012data}. Though this has not yet been fully implemented~\cite{BirdQuizzes:online}, it is evidence of the high level of effort being put in to validate the data.

For the reasons above, we are confident that the end bird feeder data provided by the citizens is accurate enough to be used in this project. Though there is data from Canada, for the scope of this project only observations from the continental United States were considered. This bird feeder data set is of such high quality that it has been used in a number of studies involving a variety of bird species~\cite{PFWPubs:online}.

\section{Related Works}

The Project FeederWatch data has the distinct advantage of being large in scale, due to the  number of participants in the study, while containing important bird species details. This allows for both broad survey style studies that include multiple bird species and also more species specific projects.

One large scale study explored the continental dominance hierarchy of various bird species across North America. Elliot T. Miller et al. used the network of citizen scientists from Project FeederWatch to discover that hierarchal standing of a species was largely predicted by the body-mass~\cite{miller2017fighting}. Another study, more similar to the focus of this project, explored the reshuffling of North American winter bird communities. This project only focused on the eastern North American area, but the team found that shifting winter climate has proved to be advantageous for smaller bird species, giving them a chance to colonize new regions~\cite{prince2015climate}.

On a more species specific scale, Barry K. Hartup et al. studied the risk factors associated with mycoplasmal conjunctivitis in eastern House Finches and made the interesting observation that the type of bird feeder may affect the risk levels of contracting the infection~\cite{hartup1998risk}. And finally there is the urgent study involving the declining Evening Grosbeak populations. This project points out that number of individuals seen at the FeederWatch sites have declined by fifty percent~\cite{bonter2008winter}. The authors are urging for more studies to be conducted in order to determine the driving factors of this decline. The data set provided by this project may aid in determining if temperature, precipitation and wind are contributing factors.


\section{Weather Underground}

Locating the proper climate data was one of the greater challenges of this study, mainly due to limited location information for each PFW observation and the large number of total observations. The climate data source needed to be able to provide accurate weather and temperature related measurements for each bird feeder observation given only the latitude, longitude and U.S. state for the location. Additionally, there are thousands of observations for each of the bird species being studied, making the total number of observations in the hundreds of thousands. 

The large amount of data processing made it a requirement for the climate data source to have a robust API through which an automated script can mine the data. This requirement made the majority of options available on the Internet unfeasible as many of the websites had specialized the human-focused web application tools for accessing their climate databases. For example, NOAA's site has the option ordering data sets by manually selecting the location and date through drop-down menus~\cite{NOAAData:online}. This process would be far too time intensive for this project. The use of the API is key to efficient access and Weather Underground is one of the few online resources that has the robust API tools.

Weather Underground (WU), like Project FeederWatch, is a citizen based data collection project. Weather Underground's network of 250,000+ personal weather stations provides accurate data for many of the major cities in the United States~\cite{WUData:online}. The best feature is that the historical and current measurements are widely accessible through their API. This makes incorporating the data mining process into the scripts very simple. Additionally, the tool is well documented which further eases the use of the API methods.

Just as with Project FeederWatch, when private citizens are providing the data great care must be taken in validating that data. Weather Underground performs a number of checks to ensure the climate data is of the highest quality possible~\cite{WUQuality:online}. An example of one such check is the temperature neighbor check. In this check, data is flagged if it differs too significantly from the neighboring temperature measurements. 

WU also employs a team of meteorologists and climatologists for the development of their proprietary weather forecasting model~\cite{WUAbout:online}. This further supports the validity of the historical and current climate measurements as they qualify for use in the forecasting applications. For the reasons highlighted above, our assumption is that the climate data collected through Weather Underground is already validated and ready for use in the study. The details of how the WU's API was used in the scripts is covered in the Methodology chapter. 

     


\section{Focus of this Project}

There are 2 large components to this project. The first portion is the climate data collection, in which climate data is appended for each PFW data point. This climate data includes the following daily measurements: maximum temperature, minimum temperature, mean temperature, precipitation level and wind speed. These parameters were deemed important by our domain expert, Professor Francis, in assessing the bird responses. Any data point for which climate data could not be found was removed from the final data set. These weather attributes are provided by Weather Underground and further details about the data collection are covered in the Methodology sections.   

The second portion of the project seeks to explore how the number of individuals seen at the bird feeders change in regards to climatic factors such as temperature, wind speed and precipitation. This will be done through analysis of the constructed dataset, which includes the original PFW data and the appended climate data. We also hypothesize that the average body mass of the bird species plays a role in the relationship between abundance and the winter temperatures.  

During the winter, food resources become important assets, as birds need more calories to maintain the appropriate body temperatures~\cite{kendeigh1949effect}. Additionally, smaller birds have higher caloric needs for temperature regulation as they do not have the advantage of a larger body mass~\cite{kendeigh1970energy}. The medium to large bird species will not need to feed as often, as they are not loosing as much heat to the surrounding environment~\cite{kendeigh1970energy}. Thus the smaller bird species are expected to visit the bird feeders more often to make-up for the calorie loss through body heating.

The bird species that are going to be focused will vary in size from small songbirds, such as the Dark-eyed Junco which has a body mass of 18 grams, to the larger Blue Jay which has a body mass of up to 100 grams~\cite{AllAboutBirds:online}. A small bird species are defined to be, for this study, any species having a mean body mass of 18--30 grams. Medium bird species are defined to have a body mass range of 40--60 grams. Large bird species are defined to have a body mass range of 70--180 grams. Additionally, the bird species of this project are known to be frequent visitors, with exception of a few species, allowing for a larger data set. The bird species selection criteria is covered in further detail in the Bird Species Selection section. 

We hypothesize that the number of the smaller bird species seen at the bird feeder will be greater when the temperature is lower than normal, as the birds are feeding more often. Additionally, we would expect to see the effect of the cold be less on the number of the larger to mid-sized bird species seen at the bird feeders.  

Lastly, wind and rain levels have been shown to impact bird health and flight behavior~\cite{erni2002wind}. It has even been observed that wet feathers can lead to torpor and eventual death~\cite{kennedy1970direct}. There are other factors at play as well, such as danger detection. Small foraging birds rely on visual cues to alert them of danger, however in a windy environment there is too much visual stimuli and thus the responsiveness of the bird may be reduced~\cite{carr2010high}. These studies indicate that it is also worthwhile to explore how bird feeding changes around feeders in accordance to the wind and rain.. 

For the above climatic attributes, our hypothesis is that as flight conditions become unfavorable during days of higher than average precipitation or wind speed, fewer individuals are spotted at the feeders. We expect that the birds will wait to visit the bird feeders when conditions are more favorable, as there is evidence that birds take shelter in trees and vegetation during stormy conditions~\cite{kennedy1970direct}.  

