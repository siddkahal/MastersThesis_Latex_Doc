\chapter{Closing Discussion}

\section{Project Summary}

To summarize, the general idea of our hypothesis is correct in regards to the responses to temperature anomalies. Many of the bird species  for this study tended to visit the bird feeders more during colder than average temperatures. There is no evidence yet of body mass determining the response and the strength of the effect of temperature anomalies. 

For the wind speed analysis, many of the species display no significant relationship. However, the general model for all species indicates that overall as the wind speed increased, more individuals visited the feeders. This general model has a significant t value of 5.054. Precipitation anomalies has an even less significant effect on the abundance of birds at the feeders when compared to wind speed. Our hypothesis on both of these climatic attributes were wrong, as we expected less birds at the feeders during periods of heavy rain or high wind. 

The Dark-eyed Junco stands out as a species because it has a strong relationship with all 3 of the climatic attributes focused on. Another stand out species in this study is the Evening Grosbeak. As mentioned in the Background sections, the population of this species is on the decline in certain areas. Our study indicates that the Evening Grosbeak is significantly impacted by temperature. However, it, along with only 2 other species, displayed a positive slope estimate value. Meaning that as the temperature gets warmer, more individuals are seen at the feeders. It is possible that the areas in which the Evening Grosbeak are declining are too cold. More definitive research needs to be done, especially to account for the migratory behavior of this species, before a conclusion can be reached.       

\section{Future Work}

This project further demonstrates the value of correlating climate data with biological data, such as the Project FeederWatch dataset. As discussed in the Methodology sections there was a limitation on how much of the PFW data set could processed. This limitation was mainly due to the Weather Underground's API, as it only allowed a set amount of calls to collect the climate data. The goal for the future is to collect climate data for all of the original PFW data set. For this study only a sample of the original data set was used. In order to append all the climate data, additional licenses or a different type of WU license must purchased in order to allow for more API calls per day.

This project has also revealed interesting species specific patterns to climate, such as with the Dark-eyed Junco and Evening Grosbeak. Though this study did not take into account any phylogeny or evolutionary history of the species, it is worthwhile to explore these ideas in the context of climate. Adding climate data to all of the PFW tuples will definitely aid in answering the next questions, and for that reason it should be among the first objectives to tackle. Once that is achieved, there is the potential to conduct many more studies. These projects can focus both on the species specific scale and the larger scale, involving many different bird species.       


