The goal of this project is to explore how 23 different bird species respond to 3 climatic attributes. These attributes are lower than average temperatures, wind speed and precipitation level. Information about the bird species and all of the data associated with them is provided by Project FeederWatch (PFW). This is a citizen based survey study that provides key information about bird species abundance through the use of backyard and community feeders. The study volunteers from across the United States and Canada monitor these bird feeders and note important information about the species such as the number of individuals seen. Other standard information is also included such as location data and date. An original data collection pipeline was developed for this study to append climate data from Weather Underground (WU) to the PFW bird feeder data. The final dataset helped to explore how exactly the birds are reacting to winter temperatures, wind speeds and rain levels. Our results indicate that birds species in general visit the bird feeders more often as temperatures dip below average. We found that the body mass of the bird plays no role in the number of visits. Birds don't seem to be significantly affected by precipitation or wind speed as our results indicate no relationship between these climatic factors and abundance at the feeders.  